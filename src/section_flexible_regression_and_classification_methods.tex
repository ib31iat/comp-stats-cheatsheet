\section{\Large Flexible regression and classification methods}\label{sec:flexible_regression_and_classification_methods}
We fight the \emph{curse of dimensionality} by making some structural assumptions (although staying with methods $g(\cdot): \mathbb{R}^p \to \mathbb{R}$ of nonparametric nature).
\subsection{Additive models}%
\label{sub:additive_models}
\begin{sectionbox}\nospacing{}
  Decompose multivariate function as $g_{\textrm add}: \mathbb{R}^p\to\mathbb{R}, x\mapsto \mu + \sum_{j=1}^pg_j(x_j)$ with $g_j(\cdot): \mathbb{R}\to\mathbb{R}$, $\mathbb{E}[g_j(X_j)]=0$.
  The zero-mean requirement for each $g_j(\cdot)$ makes the problem well posed.
  This approach is a generalization of linear models, and similarly can not model interaction terms $g_{j,k}(x_j, x_k)$.
  Due to the way they are constructed, additive linear models \emph{avoid the curse of dimensionality}.
\end{sectionbox}
\begin{sectionbox}\nospacing{}
  To fit a model, let $\mathcal{S}_j$ be a smoothing technique (e.g. \emph{Nadaraya-Watson kernel estimators}).
  Then, the \textbf{backfitting} algorithm works as follows:
  \begin{itemizenosep}
    \item Compute $\hat \mu = n^{-1}\sum_{i=1}^n Y_i$ and initialize $\hat g_j(\cdot) \coloneqq 0$.
    \item Cycle through the indices $j = 1,2,\dots,p,1,2,\dots$ and update
      $\hat g_j=\mathcal{S}_j(\vec{Y}-\hat \mu\vec{1}-\sum_{k\neq j}\hat g_k)$.
      Stop each function at convergence.
    \item Normalize the functions: $\tilde g_j(\cdot) = \hat g_j(\cdot) - n^{-1}\sum_{i=1}^n\hat g_j(X_{ij})$.
  \end{itemizenosep}
  This basically makes the algorithm repeatedly solve the 1-dimensional fitting problem.
  The algorithm may be slow but often works and can use any 1-dimensional fitting technique.
\end{sectionbox}
\begin{sectionbox}\nospacing{}
  When fitting Additive models in R with the function \verb!gam!, the smoothers $\mathcal{S}_j$ are penalized regression splines, and the degrees of freedom for each spline (i.e. each variable) will be determined through cross-validation.

  \begin{mintlinebox}{R}
    fit <- gam(Y ~ s(x1) + s(x2) + ..., data=data)
    plot(fit, pages=1, shade=TRUE) # show effect predictor
    sfsmisc::TA.plot(fit, labels="o")
  \end{mintlinebox}
\end{sectionbox}

\subsection{Multivariate adaptive regression splines}%
\label{sub:multivariate_adaptive_regression_splines}
\begin{sectionbox}\nospacing{}
  $g(\mathbf{x}) = \mu + \sum_{m=1}^{M}\beta_m h_m(\mathbf{x}) = \sum_{m=0}^{M}\beta_m h_m(\mathbf{x}) $
  Find $h \in \mathcal{M}$ functions by foward selection and pruning:

  \begin{enumeratenosep}[label=\roman*]
    \item Initialize $\mathcal{M} = \{ h_0 = 1\}, \beta_0 = \overline{Y}$
    \item For $r=1,2,..$: Find best pair (most reduction of RSS) $h_{2r-1}=h_l(\cdot)(x_j-x_{i,j})_{+}, h_{2r}=h_l(\cdot)(x_{i,j}-x_j)_{+}$, where $h_l \in \mathcal{M}$ does not already depend on $x_j$. Estimate $\beta_{2r-1}, \beta_{2r}$ by LS. Add $h_{2r-1}, h_{2r}$ to $\mathcal{M}$.
    \item Repeat until $\mathcal{M}$ large enough. Prune by repeatedly removing one function from pairs $h_{2r-1}, h_{2r}$ (least increase in RSS). Stop when GCV score is optimized.
  \end{enumeratenosep}
  \begin{mintlinebox}{R}
    require("earth"); 
    fit <- earth(formula=y~.,data=data, degree=2)
    plotmo(fit, degree2=FALSE, caption="main effects")
  \end{mintlinebox}
\end{sectionbox}	

\subsection{Neural Networks}%
\label{sub:neural_networks}
\begin{sectionbox}\nospacing{}
  $g(\mathbf{x})_k = f_0(\alpha_k + \sum_{h=1}^{q}w_{hk}\sigma(\tilde{\alpha_h} + \sum_{j=1}^{p}\tilde{w}_{jh}x_j)) \forall k=1..J$, $q$ hidden nodes, $J$ output dim. Activation $\sigma(t) = \frac{\exp t}{1 + \exp t}$. For regression, $f_0$ identity, for classification $f_0 = \sigma$ and $\mathcal{C}_{NN} = \argmax_{j} g_j(\mathbf{x})$. Many other architectures possible, e.g. including a component directly connecting input to output by linear regression.
  \begin{mintlinebox}{R}
    library(nnet); ?nnet ?ppr 
  \end{mintlinebox}
\end{sectionbox}

\subsection{Trees}%
\label{sub:trees}
\begin{sectionbox}[Classification and Regression Trees]\nospacing{}
  Let $g_{\textrm tree}(\mathbf{x}) = \sum_{r=1}^{M}\beta_r 1_{[\mathbf{x}\in \mathcal{R}_r]}$, where $\{\mathcal{R}_1,\dots, \mathcal{R}_M\}$ is a partition of $\mathbb{R}^p$.
  The function is piecewise constant.
  When a partition is given,  we can estimate $\hat{\beta}_r = \sum_{i=1}^{n}Y_i 1_{[\mathbf{x}\in \mathcal{R}_r]}/\sum_{i=1}^{n}1_{[\mathbf{x}\in \mathcal{R}_r]}$.
  For multiclass classification $\hat{\pi}_j(\mathbf{x}) = \sum_{i=1}^{n}1_{[Y_i =j]} 1_{[\mathbf{x}\in \mathcal{R}_r]}/\sum_{i=1}^{n}1_{[\mathbf{x}\in \mathcal{R}_r]}$ for $\mathbf{x} \in \mathcal{R}_r$.
  Greddy algorithm to find axes parallel partition:
  \begin{enumeratenosep}[label=\roman*]
    \item Initialize $M=1$ subset $\mathcal{P} = \{\mathcal{R} = \mathbb{R}^p\}$
    \item Split $\mathcal{R}$ at $d$ in dimension $j$, where $d$ is from the set of midpoints of observed values.
      Select $j,d$ s.t. neg. log-likelihood decrease is maximized by refinement.
    \item Apply ii) to one cell of the current partition (select like above).
      Add the resulting two cells and remove the refined one. 
    \item Iterate iii) until until specified max. partition size is achieved.
    \item Prune tree by removing leaves resulting in smalles increase in some (CV) metric.
  \end{enumeratenosep}
  The size of a tree $\mathcal{T}$ is the number of leaves (=1 + cuts).
  For some goodness-of-fit measure $\mathcal{R}(\mathcal{T})$ (e.g. SSE, NLL), the cost-complexity measure is $\mathcal{R}_{\alpha}(\mathcal{T})=\mathcal{R}(\mathcal{T}) + \alpha \cdot\text{size}(\mathcal{T}) $.
  For some $\alpha$, we thus choose $\mathcal{T}(\alpha) = \argmin_{\mathcal{T}\subset\mathcal{T}_M}\mathcal{R}_{\alpha}(\mathcal{T})$.
  The parameter $\alpha$ is chosen by CV. The 1 s.e. rule says: Choose smallest tree such that its performance is at most one standard error larger than the minimal one. 
\end{sectionbox}
\begin{mintlinebox}{R}
  library(rpart); require(rpart.plot); 
  # cp = \alpha / \mathcal{R}_{\alpha}(\mathcal{T}_{\emptyset})
  tree = rpart(y~., data=data, control=rpart.control(cp=0.0, minsplit=1)) # unregularized tree
  plotcp(tree); cps = tree$cptable 
  idx.min = which.min(cps[, "xerror"])
  std.min = cps[idx.min, "xstd"]
  cp.1se = cps[abs(cps[,"xerror"]-cps[idx.min,"xerror"])<std.min,]
  pruned.tree = prune.rpart(tree, cp=cp.1se[1, "CP"])
\end{mintlinebox}


\begin{mintlinebox}{R}
  rf <- randomForest(Boston.train, y.train, xtest=Boston.test, ytest=y.test, ntree=ntree, mtry=ncol(Boston.train))
  rf$mse[ntree] # OOB error
  rf$test$mse[ntree]
\end{mintlinebox}

\subsection{Ridge and Lasso}%
\label{sub:ridge_and_lasso}
Trade-off between $||\beta||_1$ and $||\beta||_2$ regularization, i.e. $L(\lambda_1, \lambda_2, \vec{\beta})=||\vec{Y}-X\vec{\beta}||_2^2 + (1-\alpha)/2||\vec{\beta}||_2^2+\alpha||\vec{\beta}||_1$.
$\alpha=1 \Leftrightarrow \lambda_2=1$ means full $L_2$-regularization, i.e.\ \emph{Ridge}, otherwise \emph{LASSO}.
Limited parameter size $\rightarrow$ reduced variance

\begin{mintlinebox}{R}
  ridge <- glmnet(x.train, y.train, lambda=100, alpha=1)
  ridge$beta
  ridge.cv <- cv.glmnet(x.train, y.train, lambda=grid.lambda, alpha=0, type.measure="mse")
  mean((predict(ridge.cv, newx=x.test, s="lambda.min") - y.test)^2)
  lasso.cv <- cv.glmnet(x.train, y.train, lambda=grid.lambda, alpha=1, type.measure="mse")
  mean((predict(lasso.cv, newx=x.test, s="lambda.min") - y.test)^2) 
  length(which(beta[2:..]!=0) # num non-zero predictors
\end{mintlinebox}
